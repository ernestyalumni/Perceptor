% file: ImageProcessingAndMoreCUDA.tex
% 
% github        : ernestyalumni
% gmail         : ernestyalumni 
% linkedin      : ernestyalumni 
%
% This code is open-source, governed by the Creative Common license.  Use of this code is governed by the Caltech Honor Code: ``No member of the Caltech community shall take unfair advantage of any other member of the Caltech community.'' 

\documentclass[10pt]{amsart}
\pdfoutput=1
%\usepackage{mathtools,amssymb,lipsum,caption}
\usepackage{mathtools,amssymb,caption}


\usepackage{graphicx}
\usepackage{hyperref}
\usepackage[utf8]{inputenc}
\usepackage{listings}
\usepackage[table]{xcolor}
\usepackage{pdfpages}
\usepackage{tikz}
\usetikzlibrary{graphs,matrix,arrows, positioning, shapes}

\usepackage{multicol}

\hypersetup{colorlinks=true,citecolor=[rgb]{0,0.4,0}}

\oddsidemargin=15pt
\evensidemargin=5pt
\hoffset-45pt
\voffset-55pt
\topmargin=-4pt
\headsep=5pt
\textwidth=1120pt
\textheight=595pt
\paperwidth=1200pt
\paperheight=700pt
\footskip=40pt


\newtheorem{theorem}{Theorem}
\newtheorem{corollary}{Corollary}
%\newtheorem*{main}{Main Theorem}
\newtheorem{lemma}{Lemma}
\newtheorem{proposition}{Proposition}

\newtheorem{definition}{Definition}
\newtheorem{remark}{Remark}

\newenvironment{claim}[1]{\par\noindent\underline{Claim:}\space#1}{}
\newenvironment{claimproof}[1]{\par\noindent\underline{Proof:}\space#1}{\hfill $\blacksquare$}

%This defines a new command \questionhead which takes one argument and
%prints out Question #. with some space.
\newcommand{\questionhead}[1]
  {\bigskip\bigskip
   \noindent{\small\bf Question #1.}
   \bigskip}

\newcommand{\problemhead}[1]
  {
   \noindent{\small\bf Problem #1.}
   }

\newcommand{\exercisehead}[1]
  { \smallskip
   \noindent{\small\bf Exercise #1.}
  }

\newcommand{\solutionhead}[1]
  {
   \noindent{\small\bf Solution #1.}
   }


  \title{Image Processing and More CUDA \\
\large Image Processing, Digital Image Processing, Multiple View Geometry, CUDA}

\author{Ernest Yeung \href{mailto:ernestyalumni@gmail.com}{ernestyalumni@gmail.com}}
\date{25 Feb 2022}
\keywords{C, C++, CUDA, CUDA C/C++, Image Processing, Digital Image Processing, Multiple View Geometry}

\begin{document}

\definecolor{darkgreen}{rgb}{0,0.4,0}
\lstset{language=C++,
  keywordstyle=\color{blue},
  stringstyle=\color{red},
 commentstyle=\color{darkgreen}
 }
%\lstlistoflistings

\maketitle

\tableofcontents


\begin{multicols*}{2}

\begin{abstract}

\end{abstract}


\section{How Computers Store Data}


\part{Docker}

\section{References}

This \href{https://www.cse.wustl.edu/~jain/cse570-18/ftp/m_21cdk.pdf}{lecture} on "Containers, Dockers, and Kubernetes" by Raj Jain for CSE 570-18 led me to this book: N. Poulton, "Docker Deep Dive," Oct 2017, ISBN: 9781521822807 which was bolded "Highly Recommended". Apparently there's a 2020 edition as well.

\end{multicols*}

\begin{thebibliography}{9}


\bibitem{SFR2004}
W. Richard Stevens, Bill Fenner, Andrew M. Rudoff. \textbf{UNIX Network Programming: The Sockets Networking API}. Volume 1. Third Edition. 2004. ISBN: 0-13-141155-1


\end{thebibliography}

\end{document}



